\documentclass[a4paper,12pt]{ctexart}
\usepackage[utf8]{inputenc}
\usepackage{geometry}
\usepackage{amsmath}
\usepackage{graphicx}
\usepackage{booktabs}
\usepackage{hyperref}
\usepackage{float}
\usepackage{listings}
\usepackage{xcolor}

\geometry{left=2.5cm,right=2.5cm,top=2.5cm,bottom=2.5cm}

% 代码高亮设置
\lstset{
    basicstyle=\small\ttfamily,
    columns=flexible,
    breaklines=true,
    backgroundcolor=\color{gray!10},
    frame=single,
    keywordstyle=\color{blue},
    commentstyle=\color{green!50!black},
    stringstyle=\color{red}
}

\title{\textbf{【AI入门系列】智慧笔迹:NLP新闻分类学习赛\\实验报告}}
\author{\textbf{队伍名:Attention请注意我} \\ 排名:rk3}
\date{\today}

\begin{document}

\maketitle

\section{团队成员与分工}

\begin{itemize}
    \item \textbf{郑亦航}:负责数据预处理流水线搭建、数据集特征分析(长度分布、类别不平衡分析)、数据清洗策略制定以及 HAT 模型架构的代码构建(Model Skeleton)。
    \item \textbf{李梓煊}:负责 MLM(Masked Language Modeling)预训练脚本的编写、模型训练流程的调试、Loss 函数调优以及算力资源管理。
\end{itemize}

\section{问题定义}

本次比赛的任务是“新闻文本分类”。
\begin{itemize}
    \item \textbf{输入}:匿名处理后的字符 ID 序列(空格分隔),代表一篇新闻文本。数据经过脱敏处理,无法直接看到原始汉字。
    \item \textbf{输出}:将文本分类到 14 个类别之一(如财经、体育、娱乐等),对应的 Label ID 为 0-13。
    \item \textbf{挑战}:
    \begin{enumerate}
        \item \textbf{长文本特性}:统计发现大量样本长度超过 BERT 模型的 512 限制,部分样本长达 40,000+ token。
        \item \textbf{类别不平衡}:部分类别样本数极少,最大的类别与最小的类别比例高达 42:1。
        \item \textbf{数据噪声}:存在空文本、重复文本及标签冲突的情况。
    \end{enumerate}
\end{itemize}

\section{数据分析与预处理}

\subsection{特征分析与清洗}
通过编写分析脚本,我们发现原始数据存在以下问题并进行了处理:
\begin{enumerate}
    \item \textbf{去重与冲突处理}:移除了完全重复的文本(89条)和标签冲突的样本(16条)。
    \item \textbf{长度分布}:平均长度约 900 tokens,远超 512。因此直接截断会导致严重信息丢失,必须采用长文本建模方案。
    \item \textbf{类别分布}:Label 0-3 占据主要部分,Label 13 仅有 0.45\%。我们计算了 \texttt{inverse\_sqrt} 类别权重用于 Loss 计算。
\end{enumerate}

\subsection{从原始数据到模型输入的转化过程}

\begin{enumerate}
    \item \textbf{Token ID 重映射 (Remap)}:
    原始数据的 ID 范围是 0-7549。为了给特殊 Token 预留位置,我们将所有原始 ID 偏移 +5。
    \[ ID_{new} = ID_{raw} + 5 \]
    预留位置:0=[PAD], 1=[UNK], 2=[CLS\_SEG], 3=[SEP], 4=[MASK]。

    \item \textbf{分段 (Segmentation)}:
    由于选择了层次化模型,我们需要将长文档切分为 $N$ 个片段。设定每个片段长度 $K=512$,最大片段数 $N=8$。
    \begin{itemize}
        \item 若文档长度 $L > 4096$,则截断。
        \item 若文档长度 $L < 4096$,则填充 [PAD]。
        \item \textbf{Tail Segment Pullback}:为了防止最后一个片段过短(如仅有几个 token),如果末尾片段有效长度不足阈值,我们从文档末尾向前回溯采样,保证最后一个片段也是语义丰富的。
    \end{itemize}

    \item \textbf{模型输入形状}:
    经过 `HATDataset` 处理后,一个样本被转换为张量形状:
    \[ \text{Input IDs}: [N, K] \rightarrow [8, 512] \]
    模型内部会自动添加 `[CLS]` token,最终计算形状为 $[8, 513]$。
\end{enumerate}

\section{模型设计}

鉴于文本极长的特性,传统的 BERT 模型(最大长度 512)无法处理,而 Longformer 等模型显存占用过大。经过广泛的调研后,我们选择了 \textbf{HAT (Hierarchical Attention Transformer)} 架构,具体配置为 \texttt{HAT-Interleaved-512}。

\subsection{模型选择依据}
HAT 模型通过分层机制解决长文本问题:
\begin{itemize}
    \item \textbf{计算效率}:将注意力计算复杂度从 $O(L^2)$ 降低到 $O(N \cdot K^2 + N^2)$。
    \item \textbf{全局信息交互}:通过 Segment-wise 和 Cross-segment 的交替结构,既捕捉局部细节,又捕捉全文脉络。
\end{itemize}

\subsection{模型原理}
我们的模型由 6 层 `HATLayer` 堆叠而成。每一层包含三个步骤:

\begin{enumerate}
    \item \textbf{Segment-wise Encoder (SWE)}:
    对每个 512 长度的片段独立进行 Self-Attention。
    \[ \text{SWE}(S_i) = \text{TransformerBlock}(S_i) \]
    这一步捕捉片段内的局部语义。

    \item \textbf{Cross-segment Encoder (CSE)}:
    收集所有片段的 `[CLS]` token,进行一次全局 Self-Attention。
    \[ H_{cls} = \text{Concat}([CLS]_1, [CLS]_2, ..., [CLS]_N) \]
    \[ H'_{cls} = \text{TransformerBlock}(H_{cls}) \]
    这一步让不同片段的信息进行交互。

    \item \textbf{Information Injection}:
    将交互后的全局信息 $H'_{cls}$ 重新广播(Broadcast)并加回到各片段的 token 表示中,实现全局上下文对局部的指导。
\end{enumerate}

\subsection{输出形式}
模型的最终输出取首个片段的 `[CLS]` 向量(或所有片段 `[CLS]` 的池化),通过一个线性层映射到 14 维空间:
\[ \text{Logits} = \text{Linear}(H_{final\_cls}) \in \mathbb{R}^{14} \]

\section{训练过程}

\subsection{预训练策略 (MLM)}
为了让模型适应特定的新闻数据分布,我们在有标签数据上进行了 Masked Language Modeling (MLM) 预训练。
\begin{itemize}
    \item \textbf{Mask 策略}:随机 Mask 15\% 的 token。
    \item \textbf{Loss}:计算被 Mask 位置的预测交叉熵。
    \item \textbf{效果}:通过 `mlm_train.py` 脚本,Loss 从初始的 9.12 下降至 6.89,证明模型学到了领域特征。
\end{itemize}

\subsection{Loss 函数设计}
在分类微调阶段,我们使用 \textbf{加权交叉熵损失函数 (Weighted Cross Entropy Loss)}。

\[ L = - \sum_{c=1}^{M} w_c \cdot y_{o,c} \log(p_{o,c}) \]

其中:
\begin{itemize}
    \item $M=14$ 为类别数。
    \item $y$ 为真实标签(One-hot),$p$ 为模型预测概率。
    \item \textbf{最小化含义}:交叉熵衡量的是两个概率分布的差异。最小化该 Loss 等价于最大化模型预测正确类别的似然概率,使预测分布无限接近真实分布。
    \item \textbf{权重 $w_c$}:针对类别不平衡,我们使用 `inverse_sqrt` 方法计算权重:$w_c = 1 / \sqrt{N_c}$。这使得少数类样本在梯度更新中拥有更大的话语权,防止模型偏向预测多数类。
\end{itemize}

\section{实验总结与感悟}


\end{document}