\documentclass[a4paper,12pt]{ctexart}
\usepackage[utf8]{inputenc}
\usepackage{geometry}
\usepackage{amsmath}
\usepackage{amssymb}
\usepackage{graphicx}
\usepackage{booktabs}
\usepackage{hyperref}
\usepackage{float}
\usepackage{listings}
\usepackage{xcolor}

\geometry{left=2.5cm,right=2.5cm,top=2.5cm,bottom=2.5cm}

% 代码高亮设置
\lstset{
    basicstyle=\small\ttfamily,
    columns=flexible,
    breaklines=true,
    backgroundcolor=\color{gray!10},
    frame=single,
    keywordstyle=\color{blue},
    commentstyle=\color{green!50!black},
    stringstyle=\color{red}
}

\title{\textbf{NLP新闻分类学习赛\\实验报告}}
\author{\textbf{队伍名:Attention请注意我} \\ 排名:rk3}
\date{\today}

\begin{document}

\maketitle

\section{团队成员与分工}

本项目采用“共同决策 + 分工落地”的协作机制:在方案设计阶段由两人共同调研、讨论并决策整体技术路线,
在实现阶段按职责拆分任务、各自负责落地。
为在兼顾效率与质量的前提下实现分工清晰、任务解耦,我们将工作划分为前后端两个角色,具体如下:

\begin{itemize}
    \item \textbf{郑亦航}(前端:数据与模型设计)
    \begin{itemize}
        \item 负责数据探索性分析(如文本长度分布、类别不平衡等),撰写分析报告,为后续建模与训练策略提供依据。
        \item 搭建端到端的数据预处理流水线,完成数据清洗、切分、标签重映射等脚本封装,确保全流程可复现。
        \item 设计并实现 HAT 模型基础架构,定义输入管线与关键模块接口,并完成模型骨架的单元测试。
    \end{itemize}

    \item \textbf{李梓煊}(后端:训练与部署优化)
    \begin{itemize}
        \item 实现 MLM 预训练流程,完成与数据管线、Tokenizer / ID 偏移的对接与适配,保证长文本场景下的稳定训练。
        \item 优化分类训练流程,覆盖梯度累积、混合精度训练、学习率调度等环节,在长文本场景下提升收敛速度与性能。
        \item 负责第二阶段微调,设计与调整损失函数(如不同 Loss 组合)。
        \item 实现推理部署:基于 K-fold,将各折最佳 HAT 模型用于长文档滑窗前向与窗口聚合,在此基础上进行多模型加权集成,输出最终预测结果。
    \end{itemize}
\end{itemize}


\section{问题定义}

本次比赛的任务是“新闻文本分类”。
\begin{itemize}
    \item \textbf{输入}:匿名处理后的字符 ID 序列(空格分隔),代表一篇新闻文本。数据经过脱敏处理,无法直接看到原始汉字。
    \item \textbf{输出}:将文本分类到 14 个类别之一(如财经、体育、娱乐等),对应的 Label ID 为 0-13。
    \item \textbf{挑战}:
    \begin{enumerate}
        \item \textbf{长文本结构复杂性}:许多样本长度远超 BERT 模型的 512 token 限制(部分达到 40,000+ token),且关键信息在文本中分布稀疏、跨段依赖长,分段式编码容易遗漏关键语义。
        
        \item \textbf{文本风格与领域差异}:数据来源多样,风格从新闻报道到口语评论均有,语气、表达与话题差异显著,使模型难以学习统一、稳健的特征表示。
        
        \item \textbf{类别极度不平衡}:训练集中类别数量分布长尾显著,最大与最小类别比例高达 42:1,模型在标准训练中容易偏向头部类别。
        
        \item \textbf{数据噪声显著}:存在空文本、重复文本、标签冲突、低质量内容等多种噪声,会影响训练稳定性并降低模型泛化能力。
        
        \item \textbf{计算资源与模型容量限制}:长文本 + 大规模模型导致显存压力大,限制可使用的 batch size、对抗训练策略及模型规模,使训练不够充分。
    \end{enumerate}
    
\end{itemize}

\section{数据分析与预处理}

\subsection{特征分析与清洗}
通过编写分析脚本,我们发现原始数据存在以下问题并进行了处理:
\begin{enumerate}
    \item \textbf{去重与冲突处理}:移除了完全重复的文本(89条)和标签冲突的样本(16条)。
    \item \textbf{长度分布}:平均长度约 900 tokens,远超 512。因此直接截断会导致严重信息丢失,必须采用长文本建模方案。
    \item \textbf{类别分布}:Label 0-3 占据主要部分,Label 13 仅有 0.45\%。我们计算了 \texttt{inverse\_sqrt} 类别权重用于 Loss 计算。
\end{enumerate}

\subsection{从原始数据到模型输入的转化过程}

\begin{enumerate}
    \item \textbf{Token ID 重映射 (Remap)}:
    原始数据的 ID 范围是 0-7549。为了给特殊 Token 预留位置,我们将所有原始 ID 偏移 +5。
    \[ ID_{new} = ID_{raw} + 5 \]
    预留位置:0=[PAD], 1=[UNK], 2=[CLS\_SEG], 3=[SEP], 4=[MASK]。

    \item \textbf{分段 (Segmentation)}:
    由于选择了层次化模型,我们需要将长文档切分为 $N$ 个片段。设定每个片段长度 $K=512$,最大片段数 $N=8$。
    \begin{itemize}
        \item 若文档长度 $L > 4096$,则截断。
        \item 若文档长度 $L < 4096$,则填充 [PAD]。
        \item \textbf{Tail Segment Pullback}:为了防止最后一个片段过短(如仅有几个 token),如果末尾片段有效长度不足阈值,我们从文档末尾向前回溯采样,保证最后一个片段也是语义丰富的。
    \end{itemize}

    \item \textbf{模型输入形状}:
    经过 `HATDataset` 处理后,一个样本被转换为张量形状:
    \[ \text{Input IDs}: [N, K] \rightarrow [8, 512] \]
    模型内部会自动添加 `[CLS]` token,最终计算形状为 $[8, 513]$。
\end{enumerate}

\section{模型设计}

鉴于文本极长的特性,传统的 BERT 模型(最大长度 512)无法处理,而 Longformer 等模型显存占用过大。经过广泛的调研后,我们选择了 \textbf{HAT (Hierarchical Attention Transformer)} 架构,具体配置为 \texttt{HAT-Interleaved-512}。

\subsection{模型选择依据}
HAT 模型通过分层机制解决长文本问题:
\begin{itemize}
    \item \textbf{计算效率}:将注意力计算复杂度从 $O(L^2)$ 降低到 $O(N \cdot K^2 + N^2)$。
    \item \textbf{全局信息交互}:通过 Segment-wise 和 Cross-segment 的交替结构,既捕捉局部细节,又捕捉全文脉络。
\end{itemize}

\subsection{模型原理}
我们的模型由 6 层 `HATLayer` 堆叠而成。每一层包含三个步骤:

\begin{enumerate}
    \item \textbf{Segment-wise Encoder (SWE)}:
    对每个 512 长度的片段独立进行 Self-Attention。
    \[ \text{SWE}(S_i) = \text{TransformerBlock}(S_i) \]
    这一步捕捉片段内的局部语义。

    \item \textbf{Cross-segment Encoder (CSE)}:
    收集所有片段的 `[CLS]` token,进行一次全局 Self-Attention。
    \[ H_{cls} = \text{Concat}([CLS]_1, [CLS]_2, ..., [CLS]_N) \]
    \[ H'_{cls} = \text{TransformerBlock}(H_{cls}) \]
    这一步让不同片段的信息进行交互。

    \item \textbf{Information Injection}:
    将交互后的全局信息 $H'_{cls}$ 重新广播(Broadcast)并加回到各片段的 token 表示中,实现全局上下文对局部的指导。
\end{enumerate}

\subsection{输出形式}
模型的最终输出取首个片段的 `[CLS]` 向量(或所有片段 `[CLS]` 的池化),通过一个线性层映射到 14 维空间:
\[ \text{Logits} = \text{Linear}(H_{final\_cls}) \in \mathbb{R}^{14} \]

\section{训练调优}

\subsection{预训练}
预训练阶段采用 Masked Language Modeling(MLM)让模型先学习新闻语料的语义,再进入下游分类。内容分为基本策略、关键技术与训练效果三部分:
\paragraph{基本策略}
\begin{itemize}
    \item 从预处理后的 \texttt{train.csv} 读取文本,随机打乱后按 9:1 划分训练/验证。
    \item 使用 \texttt{HATDataset} 将长文档切成最多 $N$ 个片段、每片段长度 $K{=}512$,保障长文本覆盖。
    \item \texttt{MLMDataCollator} 以 15\% 的概率对 token 进行随机 Mask(含替换/保持/置空),并对 input/label pad 对齐保持批次形状一致。
\end{itemize}
\paragraph{关键技术}
\begin{itemize}
    \item HAT MLM 结构(内部自动添加 CLS),AdamW(lr 5e-5,weight decay 0.01),线性 warmup(总步数的 6\%)+线性衰减,梯度裁剪阈值 1.0,最大 10k step,batch size 视显存调整。
    \item 正则与稳健性:启用 EMA(衰减 0.999)维护滑动平均权重以提升泛化;检测 NaN loss 并跳过异常 step。
    \item 验证与保存:每 200 step 在验证集上评估 loss / perplexity,按最优验证 loss 保存 \texttt{best\_model.pt},并周期性保存 checkpoint 便于恢复。
\end{itemize}
\paragraph{训练效果}
\begin{itemize}
    \item 在上述设置下,MLM loss 从初始 9.12 下降到 6.89,证明模型有效学习了领域分布。
\end{itemize}

\subsection{分类训练}
\paragraph{整体流程}
分类阶段采用 Stratified K-Fold(默认 5 折,种子 42)保持各折标签分布一致。每折训练 K-1 份、验证 1 份,保存 \texttt{hat\_cls\_fold\{k\}\_best.pt},最终用于多折集成。
\paragraph{数据与批处理}
\begin{itemize}
    \item 数据源为 \texttt{data/processed/train.csv}(文本 + 标签),使用 \texttt{HATDataset}(mode=train)将长文档切片为最多 $N$ 段、每段 $K{=}512$,输出不含 CLS 的 \([N,K]\)。
    \item \texttt{HATDataCollator} 打包批次,得到 \([B,N,K]\) 的 \texttt{input\_ids}/\texttt{attention\_mask} 和 \([B]\) 的标签;默认 train/eval batch 分别 64/128,dataloader worker=4。
\end{itemize}
\paragraph{模型与初始化}
\begin{itemize}
    \item 模型为 \texttt{HATInterleaved512ForClassification}(\texttt{create\_model} + \texttt{HATConfig}),内部自动添加 CLS 并输出 logits。
    \item 可加载 MLM 预训练权重 \texttt{checkpoints/mlm\_hat512/hat\_mlm\_final.pt} 进行初始化,若未提供则随机初始化。
\end{itemize}
\paragraph{损失与采样}
\begin{itemize}
    \item 损失函数通过 \texttt{create\_loss\_fn} 选择:\texttt{ce}、\texttt{smooth}(默认,label smoothing=0.05)、\texttt{focal}、\texttt{focal\_smooth}(gamma=2.0)。类别权重从 \texttt{class\_weights.npy} 加载,缓解类别不平衡。
    \item 可选 \texttt{WeightedRandomSampler} 按样本权重重采样,进一步平衡长尾。
\end{itemize}
\paragraph{训练策略}
\begin{itemize}
    \item 优化器 AdamW(lr 3e-5,weight decay 0.01,betas 0.9/0.999,eps 1e-8),线性 warmup(6\% 总步数)后线性衰减;默认 5 个 epoch,总步数 = batch 数 × epoch。
    \item 梯度裁剪阈值 1.0;可选 AMP 混合精度加速与省显存;可选 EMA(decay 0.9999)维护滑动平均权重,验证时优先用 EMA 模型。
    \item 每 50 step 打印训练日志(loss、lr、吞吐),按 step 推进 scheduler;遇 NaN loss 跳过该 step。
\end{itemize}
\paragraph{验证与早停}
\begin{itemize}
    \item 每个 epoch 结束在当前折验证集评估 loss / accuracy / macro-F1,使用 EMA 权重(若启用);选取 macro-F1 最优的权重保存为 \texttt{hat\_cls\_fold\{k\}\_best.pt},并可同时保存原始权重以便继续训练。
    \item 支持 early stopping(基于 macro-F1,无提升达 patience 即停止),每折结束后清理显存再训练下一折。
\end{itemize}
\paragraph{结果与使用}
\begin{itemize}
    \item 记录各折最佳 macro-F1 及路径,给出均值/方差作为稳健性指标;推理阶段可加载全部折的最佳权重做集成提升稳定性。
\end{itemize}

\subsection{第二阶段微调}
\paragraph{目标与流程}
在 Stage1 基础上,对每个 \texttt{hat\_cls\_fold\{k\}\_best.pt} 进行小学习率二次微调;若 Stage2 的最佳 F1 不低于 Stage1,则替换为 \texttt{hat\_cls\_fold\{k\}\_stage2\_best.pt},否则沿用原模型。
\paragraph{数据与增强}
\begin{itemize}
    \item 仍使用 \texttt{train.csv} 全量数据与同一 K 折划分;验证集缓存切分,训练集关闭缓存。
    \item 自定义 \texttt{Stage2HATDataset} 引入 token 级随机滑窗起点(stride 默认 128),仅训练时开启以增加样本多样性;\texttt{HATDataCollator} 负责批处理。
\end{itemize}
\paragraph{训练与正则}
\begin{itemize}
    \item 小学习率 AdamW(lr 1e-5,weight decay 0.01),warmup 5\%,epoch 1\textasciitilde{}2,batch 64/128,梯度裁剪 1.0,支持 AMP。
    \item 可选 EMA(decay 0.9999)用于验证/保存;可选 R-Drop(一批双前向+KL,系数 0.5)提升一致性。
    \item 可选 FGM 对抗训练(embedding 层 \(\epsilon{=}0.5\),对抗损失权重 1.0),与 R-Drop/AMP 兼容。
\end{itemize}
\paragraph{验证与保存}
\begin{itemize}
    \item 每个 epoch 结束在当前折验证集计算 loss / Acc / macro-F1;仅当 Stage2 F1 \(\ge\) Stage1 最优时保存 Stage2 权重,否则保留 Stage1。
    \item 训练过程中按 step 记录 CE/KL 及 lr/吞吐,支持 early stopping(默认 patience=1)。
\end{itemize}
\paragraph{结果与分析}
\begin{itemize}
    \item 实测 Stage2 未能显著超过 Stage1(微调后 F1 持平或略降),最终沿用 Stage1 权重。
    \item 可能原因:Stage1 已接近性能上限,小学习率+少量 step 难以带来有效优化;R-Drop/FGM 等正则增加噪声,收益被短程训练抵消;随机滑窗带来的分布漂移在有限步内未收敛。
\end{itemize}

\section{推理部署}
\paragraph{整体流程}
推理阶段通过 \texttt{infer\_kfold.py} 自动收集各折模型(优先使用 Stage2 \texttt{hat\_cls\_fold\{k\}\_stage2\_best.pt},缺失则回退 Stage1 \texttt{hat\_cls\_fold\{k\}\_best.pt}),拼接为逗号分隔的路径后调用 \texttt{infer.py} 完成滑窗 + 多模型集成推理,生成提交文件。
\paragraph{数据与滑窗}
\begin{itemize}
    \item \textbf{Dataset/Collator}:\texttt{InferenceDataset} 对每条样本保留 doc\_id 并在 collator 中展开全部滑窗;\texttt{InferenceCollator} 将每个窗口 pad 到固定 \([N,K]\)(默认 8×512),再堆叠为 \([B_{total},N,K]\)。
    \item \textbf{窗口 TTA}:支持 token offset 列表(如 \texttt{--window-tta-offsets 0,64}),为同一文档生成多视角窗口集合,提升鲁棒性。
\end{itemize}
\paragraph{模型加载与加权}
\begin{itemize}
    \item 多 checkpoint 由 \texttt{load\_models} 加载;若 ckpt 内含 \texttt{val\_macro\_f1},则按该分数归一化为模型权重,否则使用均匀权重。
    \item 推理可在 GPU/CPU 运行,支持 \texttt{--num-workers} 控制 DataLoader 并行。
\end{itemize}
\paragraph{推理技术要点}
\begin{itemize}
    \item \textbf{滑窗聚合}:对同一文档的多窗口 logits 先做窗口级聚合(\texttt{mean} / \texttt{max} / \texttt{mean\_conf} 置信度加权均值),得到文档级 logits。
    \item \textbf{多模型集成}:在模型维度再聚合(\texttt{logits\_avg} / \texttt{prob\_avg} / \texttt{voting},均支持 \texttt{*\_weighted} 按验证分数加权),默认 \texttt{prob\_avg\_weighted}(在 \texttt{infer\_kfold.py} 中)或 \texttt{logits\_avg}(在 \texttt{infer.py} 中)。
    \item \textbf{MC Dropout}:\texttt{--mc-dropout-runs} $>1$ 时仅开启 Dropout 层训练态,多次前向取均值,提供不确定性平滑。
    \item \textbf{阈值策略}:支持二分类阈值 \texttt{--decision-threshold},或多类逐类阈值 \texttt{--class-thresholds};若提供验证集,可用 \texttt{--tune-class-thresholds} + 网格搜索自动找统一阈值。
\end{itemize}
\paragraph{校验与输出}
\begin{itemize}
    \item 可选验证集 sanity check:在 val 集上跑完整推理链路评估 Macro-F1/Acc,验证聚合与阈值配置的合理性。
    \item 默认输出天池提交格式 CSV(可指定 sample\_submit 校验列名),并可选保存聚合 logits 方便分析。
\end{itemize}
\paragraph{实际表现}
\begin{itemize}
    \item 在本任务中,Stage2 模型未优于 Stage1,\texttt{infer\_kfold.py} 会自动优先使用存在的 Stage2 ckpt;若缺失则回退 Stage1,保证推理流程稳定。
\end{itemize}

\section{实验总结与感悟}

这次做长文本分类,最大的收获是“先把链路跑通,再慢慢抠细节”。一开始总想一步到位上所有花活,但显存、时长都把我们掐得死死的。后来改成先用最小可行方案通关,再逐块替换成 HAT、滑窗、集成,反而节奏更稳。

长文本处理的两个坑印象最深:其一是切分策略,末尾片段太短会把信息丢光,Tail Segment Pullback 确实帮了大忙;其二是类别不平衡,class weight + sampler 组合比我们原来只用一种手段稳得多。另一个小教训是 Stage2 微调:理论上应该更好,但短程微调加随机滑窗带来的分布漂移,把收益抵消了,提醒我们“没必要的套路就别硬上”。

调参心得:学习率、warmup 比想象中敏感,尤其是带 EMA 时,稍大的 lr 会导致 EMA 权重“跟不上节奏”,验证集抖动明显;梯度裁剪和 AMP 基本是白送的稳定性提升,可以默认开。日志和 checkpoint 也要勤做,遇到 NaN 能快速回溯。

推理阶段,滑窗聚合 + 多模型加权的组合挺香,但要注意推理时长与提交截止的平衡;我们最后保留了 window TTA 的接口,但默认不开,以免超时。整体体验下来,K 折 + 加权集成是当前资源约束下最稳妥的解。

如果有更多时间,想补的方向有:更长的预训练步数(或更好的预训练数据),尝试对比 Hyena / Mamba 这类长序列模型,和更系统的阈值调优。但至少目前的方案在可靠性和复现性上已经可落地。
\end{document}